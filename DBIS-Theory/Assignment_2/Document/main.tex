\documentclass{article}
\usepackage[english]{babel}
\usepackage[utf8]{inputenc}
\usepackage{amsmath,amssymb}
\usepackage{parskip}
\usepackage{graphicx}
\usepackage{dsfont}
\usepackage{dsfont}
\usepackage{relsize}
\usepackage{array}
\newcommand{\bigsigma}{\makebox{\Huge\ensuremath{\sigma}}}
\newcommand{\bigpi}{\makebox{\Huge\ensuremath{\Pi}}}
\newcolumntype{C}[1]{>{\centering\let\newline\\\arraybackslash\hspace{0pt}}m{#1}}
\usepackage[top=2.5cm, left=3cm, right=3cm, bottom=4.0cm]{geometry}
\usepackage[table]{xcolor}
\usepackage[utf8]{inputenc}
\usepackage{textcomp}
\usepackage[utf8]{inputenc}
\usepackage{amsmath}
\usepackage{amssymb}
\usepackage{xcolor}
\usepackage{listings}
\usepackage{xstring}
\usepackage{graphicx}
\usepackage[export]{adjustbox}

\definecolor{dkgreen}{rgb}{0,0.6,0}
\definecolor{ltgray}{rgb}{0.5,0.5,0.5}

\makeatletter
\newif\ifcolname
\colnamefalse

\def\keywordcheck{%
\IfStrEq*{\the\lst@token}{select}{\global\colnametrue}{}%
\IfStrEq*{\the\lst@token}{where}{\global\colnametrue}{}%
\IfStrEq*{\the\lst@token}{from}{\global\colnamefalse}{}%
\color{blue}%
}
\def\setidcolor{%
\ifcolname\color{purple}\else\color{black}\fi%
}
\makeatother

\lstset{%
    backgroundcolor=\color{white},
    basicstyle=\footnotesize,
    breakatwhitespace=false,
    breaklines=true,
    captionpos=b,
    commentstyle=\color{dkgreen},
    deletekeywords={...},
    escapeinside={\%*}{*)},
    extendedchars=true,
    frame=single,
    keepspaces=true,
    language=SQL,
    otherkeywords={is},
    morekeywords={*,modify,MODIFY,...},
    keywordstyle=\keywordcheck,
    identifierstyle=\setidcolor,
    numbers=left,
    numbersep=15pt,
    numberstyle=\tiny,
    rulecolor=\color{ltgray},
    showspaces=false,
    showstringspaces=false, 
    showtabs=false,
    stepnumber=1,
    tabsize=4,
    title=\lstname
}

\newcommand{\tablespace}{\\[1.25mm]}
\newcommand\Tstrut{\rule{0pt}{2.6ex}}         % = `top' strut
\newcommand\tstrut{\rule{0pt}{2.0ex}}         % = `top' strut
\newcommand\Bstrut{\rule[-0.9ex]{0pt}{0pt}}   % = `bottom' strut
\title{Assignment-2 CS303}
\author{Shashank P \\ 200010048}
\date{\today}

\begin{document}
\maketitle




\section{Problem 1}
Suppose that we have a relation marks(ID, score) and we wish to assign grades to students based on the
score as follows: grade F if score < 40,grade C if 40 $\leq$ score $<$ 60, grade B if 60 $\leq$ score < 80, and grade
A if 80 $\leq$ score. Write SQL queries to do the following:
\subsection{Part (a)}
Display the grade for each student, based on the marks relation.
\begin{lstlisting}[language=sql]
select ID, score,
  case 
      when score>=80 then 'A'
      when score>=60 then 'B'
      when score>=40 then 'C'
      else 'F'
  end as grade
from marks
\end{lstlisting}

\subsection{Part (b)}
Display the grade for each student, based on the marks relation.
\begin{lstlisting}[language=sql]
select case 
    when score>=80 then 'A'
    when score>=60 then 'B'
    when score>=40 then 'C'
    else 'F'
  end as grade, count(*) as grade_count
from marks
group by grade
\end{lstlisting}

\section{Problem 2}
Using tables given in the question, write SQL queries for the following:
\subsection{Part (a)}
Give all employees of “First Bank Corporation” a 10 percent raise.
\begin{lstlisting}[language=sql]
update works
set salary = salary*1.1
where company_name='First Bank Corporation'
\end{lstlisting}

\subsection{Part (b)}
Give all managers of “First Bank Corporation” a 10 percent raise.
\begin{lstlisting}[language=sql]
update works
set salary = salary*1.1
where employee_name in 
    (select distinct M.manager_name
    from works as W inner join manages as M
    on W.employee_name=M.manager_name
    where W.company_name='First Bank Corporation')
\end{lstlisting}

\subsection{Part (c)}
Delete all tuples in the works relation for employees of “Small Bank Corporation”.
\begin{lstlisting}[language=sql]
delete from works
where company_name='Small Bank Corporation'
\end{lstlisting}

\subsection{Part (d)}
Delete all tuples in the works relation for employees of “Small Bank Corporation”.
\begin{lstlisting}[language=sql]
delete from works
where company_name='Small Bank Corporation'
\end{lstlisting}

\end{document}
