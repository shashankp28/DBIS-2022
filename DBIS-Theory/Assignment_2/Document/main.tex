\documentclass{article}
\usepackage[english]{babel}
\usepackage[utf8]{inputenc}
\usepackage{amsmath,amssymb}
\usepackage{parskip}
\usepackage{graphicx}
\usepackage{dsfont}
\usepackage{dsfont}
\usepackage{relsize}
\usepackage{array}
\newcommand{\bigsigma}{\makebox{\Huge\ensuremath{\sigma}}}
\newcommand{\bigpi}{\makebox{\Huge\ensuremath{\Pi}}}
\newcolumntype{C}[1]{>{\centering\let\newline\\\arraybackslash\hspace{0pt}}m{#1}}
\usepackage[top=2.5cm, left=3cm, right=3cm, bottom=4.0cm]{geometry}
\usepackage[table]{xcolor}
\usepackage[utf8]{inputenc}
\usepackage{textcomp}
\usepackage[utf8]{inputenc}
\usepackage{amsmath}
\usepackage{amssymb}
\usepackage{xcolor}
\usepackage{listings}
\usepackage{xstring}
\usepackage{graphicx}
\usepackage[export]{adjustbox}

\definecolor{dkgreen}{rgb}{0,0.6,0}
\definecolor{ltgray}{rgb}{0.5,0.5,0.5}

\makeatletter
\newif\ifcolname
\colnamefalse

\def\keywordcheck{%
\IfStrEq*{\the\lst@token}{select}{\global\colnametrue}{}%
\IfStrEq*{\the\lst@token}{where}{\global\colnametrue}{}%
\IfStrEq*{\the\lst@token}{from}{\global\colnamefalse}{}%
\color{blue}%
}
\def\setidcolor{%
\ifcolname\color{purple}\else\color{black}\fi%
}
\makeatother

\lstset{%
    backgroundcolor=\color{white},
    basicstyle=\footnotesize,
    breakatwhitespace=false,
    breaklines=true,
    captionpos=b,
    commentstyle=\color{dkgreen},
    deletekeywords={...},
    escapeinside={\%*}{*)},
    extendedchars=true,
    frame=single,
    keepspaces=true,
    language=SQL,
    otherkeywords={is},
    morekeywords={*,modify,MODIFY,...},
    keywordstyle=\keywordcheck,
    identifierstyle=\setidcolor,
    numbers=left,
    numbersep=15pt,
    numberstyle=\tiny,
    rulecolor=\color{ltgray},
    showspaces=false,
    showstringspaces=false, 
    showtabs=false,
    stepnumber=1,
    tabsize=4,
    title=\lstname
}

\newcommand{\tablespace}{\\[1.25mm]}
\newcommand\Tstrut{\rule{0pt}{2.6ex}}         % = `top' strut
\newcommand\tstrut{\rule{0pt}{2.0ex}}         % = `top' strut
\newcommand\Bstrut{\rule[-0.9ex]{0pt}{0pt}}   % = `bottom' strut
\title{Assignment-2 CS303}
\author{Shashank P \\ 200010048}
\date{\today}

\begin{document}
\maketitle




\section{Problem 1}
Suppose that we have a relation marks(ID, score) and we wish to assign grades to students based on the
score as follows: grade F if score < 40,grade C if 40 $\leq$ score $<$ 60, grade B if 60 $\leq$ score < 80, and grade
A if 80 $\leq$ score. Write SQL queries to do the following:
\subsection{Part (a)}
Display the grade for each student, based on the marks relation.
\begin{lstlisting}[language=sql]
select ID, score,
  case 
      when score>=80 then 'A'
      when score>=60 then 'B'
      when score>=40 then 'C'
      else 'F'
  end as grade
from marks
\end{lstlisting}

\subsection{Part (b)}
Display the grade for each student, based on the marks relation.
\begin{lstlisting}[language=sql]
select case 
    when score>=80 then 'A'
    when score>=60 then 'B'
    when score>=40 then 'C'
    else 'F'
  end as grade, count(*) as grade_count
from marks
group by grade
\end{lstlisting}

\newpage
\section{Problem 2}
Using tables given in the question, write SQL queries for the following:
\subsection{Part (a)}
Give all employees of “First Bank Corporation” a 10 percent raise.
\begin{lstlisting}[language=sql]
update works
set salary = salary*1.1
where company_name='First Bank Corporation'
\end{lstlisting}

\subsection{Part (b)}
Give all managers of “First Bank Corporation” a 10 percent raise.
\begin{lstlisting}[language=sql]
update works
set salary = salary*1.1
where employee_name in 
    (select distinct M.manager_name
    from works as W inner join manages as M
    on W.employee_name=M.manager_name
    where W.company_name='First Bank Corporation')
\end{lstlisting}

\subsection{Part (c)}
Delete all tuples in the works relation for employees of “Small Bank Corporation”.
\begin{lstlisting}[language=sql]
delete from works
where company_name='Small Bank Corporation'
\end{lstlisting}

\subsection{Part (d)}
Find all employees in the database who live in the same cities as the companies for which they work.
\begin{lstlisting}[language=sql]
select distinct employee.employee_name
from employee, works, company
where employee.city=company.city
    and employee.employee_name=works.employee_name
    and works.company_name=company.company_name
\end{lstlisting}

\newpage
\subsection{Part (e)}
Find all employees in the database who live in the same cities and on the same streets as do their
managers.
\begin{lstlisting}[language=sql]
with manager(manager_name, street, city) as
  (select employee_name, street, city
  from employee
  where employee_name in
      (select distinct T.manager_name
      from manages as T))
select distinct E.employee_name
from employee as E, manages as Ms, manager as Mr
where E.employee_name=Ms.employee_name
  and Mr.manager_name=Ms.manager_name
  and E.street=Mr.street
  and E.city=Mr.city
\end{lstlisting}

\subsection{Part (f)}
Find all employees who earn more than the average salary of all employees of their company.
\begin{lstlisting}[language=sql]
with avg_salary(c_name, val) as
  (select company_name, avg(salary)
  from works
  group by company_name)
select distinct W.employee_name
from works as W inner join avg_salary
on W.company_name=avg_salary.c_name 
where W.salary>avg_salary.val
\end{lstlisting}

\subsection{Part (g)}
Find the company that has the smallest payroll.
\begin{lstlisting}[language=sql]
with salary_sum(c_name, val) as
  (select company_name, sum(salary)
  from works
  group by company_name)
with min_sum(val) as
  (select min(val)
  from salary_sum)
select distinct salary_sum.c_name
from salary_sum, min_sum
where salary_sum.val=min_sum.val
\end{lstlisting}

\newpage
\subsection{Part (h)}
Find the company that has the most employees.
\begin{lstlisting}[language=sql]
with pay_counts(c_name, val) as
  (select company_name, count(distinct employee_name)
  from works
  group by company_name)
with max_count(val) as
  (select max(val)
  from pay_counts)
select distinct pay_counts.c_name
from pay_counts, max_count
where pay_counts.val=max_count.val
\end{lstlisting}

\subsection{Part (i)}
Find those companies whose employees earn a higher salary, on average, than the average salary at
“First Bank Corporation”.
\begin{lstlisting}[language=sql]
with FBC_avg(val) as
  (select avg(salary)
  from works
  where company_name='First Bank Corporation')
with avg_salary(c_name, val) as
  (select company_name, avg(salary)
  from works
  group by company_name)
select distinct c_name
from avg_salary, FBC_avg
where avg_salary.val>FBC_avg.val
\end{lstlisting}

\subsection{Part (j)}
Modify the database so that “Jones” now lives in “Newtown”.
\begin{lstlisting}[language=sql]
update employee
set city='Newtown'
where employee_name='Jones'
\end{lstlisting}

\newpage
\subsection{Part (k)}
Give all managers of “First Bank Corporation” a 10 percent raise unless the salary becomes greater
than \$100,000; in such cases, give only a 3 percent raise.
\begin{lstlisting}[language=sql]
update works
set salary = case
    when salary*1.1>100000 then salary*1.03
    else salary*1.1
    end
where employee_name in 
    (select distinct M.manager_name
    from works as W inner join manages as M
    on W.employee_name=M.manager_name
    where W.company_name='First Bank Corporation')
\end{lstlisting}

\section{Problem 3}
Write a query to get the list of users who took a training lesson more than once in the same day, grouped by
user and training lesson, each ordered from the most recent lesson date to oldest date.
\begin{lstlisting}[language=sql]
select *
from users natural join 
    (select user_id, training_id, training_date
    from training_details
    group by user_id, training_id, training_date
    having count(*)>1)
order by training_date desc
\end{lstlisting}

\section{Problem 4}
What is the meaning of the query given below? 
\begin{lstlisting}[language=sql]
  SELECT * FROM runners WHERE id NOT IN (SELECT winner_id FROM races)
\end{lstlisting}
The \textbf{meaning of the query} is to get the details of all those runners who did not win in any of the races.
But it will return as empty set as \textbf{NOT IN} key word is used,
which will return an \textbf{empty set} if any walue in the sub query is \textbf{NULL}.

\section{Problem 5}
Consider books and publishers table. A publisher may have zero or many books while a book may belong to zero or one publisher. The relationship
between the books table and the publishers table is zero-to-many.
\newpage
\subsection{Part (a)}
A query which will return information about books with publishers, irrespective of whether a book
has associated publishers or not.
\begin{lstlisting}[language=sql]
select *
from books left outer join publishers
on books.publisher_id=publishers.publisher_id
\end{lstlisting}

\subsection{Part (b)}
A query which will return information about books with publishers, irrespective of whether the
publisher has any published books or not.
\begin{lstlisting}[language=sql]
select *
from books right outer join publishers
on books.publisher_id=publishers.publisher_id
\end{lstlisting}

\end{document}
