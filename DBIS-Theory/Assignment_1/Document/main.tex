\documentclass{article}

\usepackage[english]{babel}
\usepackage[utf8]{inputenc}
\usepackage{amsmath,amssymb}
\usepackage{parskip}
\usepackage{graphicx}
\usepackage{dsfont}
\usepackage{dsfont}
\usepackage{relsize}
\usepackage{array}
\newcommand{\bigsigma}{\makebox{\Huge\ensuremath{\sigma}}}
\newcommand{\bigpi}{\makebox{\Huge\ensuremath{\Pi}}}
\newcolumntype{C}[1]{>{\centering\let\newline\\\arraybackslash\hspace{0pt}}m{#1}}
\usepackage[top=2.5cm, left=3cm, right=3cm, bottom=4.0cm]{geometry}
\usepackage[table]{xcolor}

\newcommand{\tablespace}{\\[1.25mm]}
\newcommand\Tstrut{\rule{0pt}{2.6ex}}         % = `top' strut
\newcommand\tstrut{\rule{0pt}{2.0ex}}         % = `top' strut
\newcommand\Bstrut{\rule[-0.9ex]{0pt}{0pt}}   % = `bottom' strut
\title{Assignment-1 CS303}
\author{Shashank P \\ 200010048}
\date{\today}

\begin{document}
\maketitle




\section{Problem 1}
\textbf{The Schema of the table is as shown below}
\begin{center}
    \textbf{Branch} (branch\_name, branch\_city, assets) \\ 
    \textbf{customer} (customer\_name, customer\_street, customer\_city) \\ 
    \textbf{loan} (loan\_number, branch\_name, amount) \\ 
    \textbf{borrower} (customer\_name, loan\_number) \\ 
    \textbf{account} (account\_number, branch\_name, balance) \\ 
    \textbf{depositor} (customer\_name, account\_number) 
\end{center}


\subsection{Sub Question (a)}

\subsubsection{Part (i)}
\hspace{\parindent} To find the names of braches present in \textbf{Chicago}, we
impose a condition on \textit{branch\_city} using table \textbf{Branch}. Projection
can be used to get the names.
\begin{equation}
  \Large \bigpi_{branch\_name} (\bigsigma_{branch\_name = "Chicago"}(Branch))
\end{equation}

\subsubsection{Part (ii)}
\hspace{\parindent} To solve this we take the cross product of 
\textbf{borrower} and \textbf{loan} tables and join them based on
appropriate conditions. Projection is used to select the customer names.
\begin{equation}
  \Large 
    \begin{split}
       & P_1 \equiv  borrower.loan\_number = loan.loan\_number \\
       & P_2 \equiv  loan.branch\_name = "Downtown" \\
       & \bigpi_{borrower.customer\_name}  (\bigsigma_{P_1 \wedge \ P_2}(borrower \times loan))
    \end{split}
\end{equation}
 
\newpage


\subsection{Sub Question (b)}
In this question I am assuming the following

\begin{itemize}
  \item \textit{customer\_name} in \textbf{Customer} table are unique.
  \item Multiple Customers can be associated with the same loan (Joint Loans).
  \item One customer can take up multiple loans. 
  \item Multiple people can open one acccount (Joint Account).
  \item One person can have only a single account number (As bank is the same in this case).
\end{itemize}

  \begin{table}[ht]
    \centering
    \begin{center}
    \begin{tabular}{||C{3cm}||C{3cm}||C{3cm}||}
    \hline  
    \hline
    \textbf{Table} & \textbf{Primary Key}  & \textbf{Foreign Key}\\
    \hline \hline

    Branch & branch\_name  & -- \\
    \hline \hline

    customer & customer\_name  & -- \\
    \hline \hline

    loan & loan\_number  & branch\_name ref. Branch \\
    \hline \hline

    borrower & (customer\_name, loan\_number)  & customer\_name ref. customer loan\_number ref. loan \\
    \hline \hline

    account & account\_number  & branch\_name ref. Branch \\
    \hline \hline

    depositor & customer\_name  & customer\_name ref. customer acccount\_number ref. account \\
    \hline \hline

    \end{tabular}
  \end{center}
  \caption{Keys in the given Schema}
  \end{table}

\subsection{Sub Question (c)}
\subsubsection{Part (i)}
\hspace{\parindent} First use the select operator then project the \textit{loan\_numebr} on \textbf{loan} table.


\begin{equation}
  \Large \bigpi_{loan\_number} (\bigsigma_{amount>10000}(loan))
\end{equation}

\subsubsection{Part (ii)}
\hspace{\parindent} We first join \textit{account} and \textit{depositer}
tables. Then we use the select and project operator.


\begin{equation}
  \Large 
    \begin{split}
       & P_1 \equiv  account.account\_number = depositor.account\_number \\
       & P_2 \equiv  account.balance > 6000 \\
       & \bigpi_{depositor.customer\_name}  (\bigsigma_{P_1 \wedge \ P_2}(account \times depositor))
    \end{split}
\end{equation}

\subsubsection{Part (iii)}
\hspace{\parindent} Similar to previous question, but we have another condition on branch\_name.


\begin{equation}
  \Large 
    \begin{split}
       & P_1 \equiv  account.account\_number = depositor.account\_number \\
       & P_2 \equiv  account.balance > 6000 \\
       & P_3 \equiv  account.branch\_name = "Uptown" \\
       & \bigpi_{depositor.customer\_name}  (\bigsigma_{P_1 \ \wedge \ P_2 \ \wedge \ P_3 }(account \times depositor))
    \end{split}
\end{equation}

\section{Problem 2}
\subsection{Part (i)}
Selecting users whose age is more that 25.
\begin{table}[ht]
  \centering
  \begin{center}
  \begin{tabular}{|C{4cm}|}
  \hline
  \textbf{Name} \\  \hline 
  Victor \\   \hline
  Jane \\   \hline
  \end{tabular}
\end{center}
\caption{Output}
\end{table}


\subsection{Part (ii)}
Select users whose Id is greater than 2 or whose age is not 31.
\begin{table}[ht]
  \centering
  \begin{center}
  \begin{tabular}{|C{2cm}|C{2cm}|C{2cm}|C{2cm}|C{3cm}|C{2cm}|}
  \hline
  \textbf{Id} & \textbf{Name}  & \textbf{Age} & \textbf{Gender} & \textbf{OccupationId} & \textbf{CityId} \\  \hline 
  1      & John        & 25   & Male  & 1 & 3 \\   \hline
  2      & Sara        & 20   & Female  & 3 & 4 \\   \hline
  3      & Victor        & 31   & Male  & 2 & 5 \\   \hline
  4      & Jane        & 27   & Female  & 1 & 3 \\   \hline
  \end{tabular}
\end{center}
\caption{Output}
\end{table}

\newpage
\subsection{Part (iii)}
Join tables \textit{User} and \textit{Occupation}.
\begin{table}[ht]
  \centering
  \begin{center}
  \begin{tabular}{|C{1cm}|C{2cm}|C{1cm}|C{2cm}|C{2cm}|C{2cm}|C{2cm}|C{2cm}|}
  \hline
  \textbf{Id} & \textbf{Name}  & \textbf{Age} & \textbf{Gender} & \textbf{OccupationId} & \textbf{CityId} & \textbf{OccupationId} & \textbf{OccupationName} \\  \hline 
  1      & John        & 25   & Male  & 1 & 3 & 1 & Software Engineer \\   \hline
  2      & Sara        & 20   & Female  & 3 & 4 & 3 & Pharmacist \\   \hline
  3      & Victor        & 31   & Male  & 2 & 5 & 2 & Accountant \\   \hline
  4      & Jane        & 27   & Female  & 1 & 3 & 3 & Software Engineer \\   \hline
  \end{tabular}
\end{center}
\caption{Output}
\end{table}

\subsection{Part (iv)}
Natural join tables \textit{User}, \textit{Occupation} and \textit{City}.

\begin{table}[ht]
  \centering
  \begin{center}
  \begin{tabular}{|C{1cm}|C{2cm}|C{1cm}|C{2cm}|C{2cm}|C{2cm}|C{2cm}|C{2cm}|C{2cm}|}
  \hline
  \textbf{Id} & \textbf{Name}  & \textbf{Age} & \textbf{Gender} & \textbf{OccupationId} & \textbf{CityId} & \textbf{OccupationName} & \textbf{CityName} \\  \hline 
  1      & John        & 25   & Male  & 1 & 3 & Software Engineer & Boston \\   \hline
  2      & Sara        & 20   & Female  & 3 & 4 & Pharmacist & New York \\   \hline
  3      & Victor        & 31   & Male  & 2 & 5 & Accountant & Toronto \\   \hline
  4      & Jane        & 27   & Female  & 1 & 3 & Software Engineer & Boston \\   \hline
  \end{tabular}
\end{center}
\caption{Output}
\end{table}

\subsection{Part (v)}
Name and gender of users who live in Boston.

\begin{table}[ht]
  \centering
  \begin{center}
  \begin{tabular}{|C{3cm}|C{3cm}|}
  \hline
  \textbf{Name} & \textbf{Gender} \\ \hline 
  John        & Male   \\   \hline
  Jane        & Female \\   \hline
  \end{tabular}
\end{center}
\caption{Output}
\end{table}

\end{document}
