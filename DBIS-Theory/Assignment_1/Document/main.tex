\documentclass{article}

\usepackage[english]{babel}
\usepackage[utf8]{inputenc}
\usepackage{amsmath,amssymb}
\usepackage{parskip}
\usepackage{graphicx}
\usepackage{dsfont}
\usepackage{dsfont}
\usepackage{relsize}
\newcommand{\bigsigma}{\makebox{\Huge\ensuremath{\sigma}}}
\newcommand{\bigpi}{\makebox{\Huge\ensuremath{\Pi}}}

% Margins
\usepackage[top=2.5cm, left=3cm, right=3cm, bottom=4.0cm]{geometry}
% Colour table cells
\usepackage[table]{xcolor}

% Get larger line spacing in table
\newcommand{\tablespace}{\\[1.25mm]}
\newcommand\Tstrut{\rule{0pt}{2.6ex}}         % = `top' strut
\newcommand\tstrut{\rule{0pt}{2.0ex}}         % = `top' strut
\newcommand\Bstrut{\rule[-0.9ex]{0pt}{0pt}}   % = `bottom' strut

%%%%%%%%%%%%%%%%%
%     Title     %
%%%%%%%%%%%%%%%%%
\title{Assignment-1 CS303}
\author{Shashank P \\ 200010048}
\date{\today}

\begin{document}
\maketitle

%%%%%%%%%%%%%%%%%
%   Problem 1   %
%%%%%%%%%%%%%%%%%
\section{Problem 1}
\textbf{The Schema of the table is as shown below}
\begin{center}
    \textbf{Branch} (branch\_name, branch\_city, assets) \\ 
    \textbf{customer} (customer\_name, customer\_street, customer\_city) \\ 
    \textbf{loan} (loan\_number, branch\_name, amount) \\ 
    \textbf{borrower} (customer\_name, loan\_number) \\ 
    \textbf{account} (account\_number, branch\_name, balance) \\ 
    \textbf{depositor} (customer\_name, account\_number) 
\end{center}
\subsection{Sub Question (a)}
\subsubsection{Part (i)}
\hspace{\parindent} To find the names of braches present in \textbf{Chicago}, we
impose a condition on \textit{branch\_city} using table \textbf{Branch}. Projection
can be used to get the names.

\begin{equation}
  \Large \bigpi_{branch\_name} (\bigsigma_{branch\_name="Chicago"}(Branch))
\end{equation}

\subsubsection{Part (ii)}
\hspace{\parindent} To solve this we take the cross product of 
\textbf{borrower} and \textbf{loan} tables and join them based on
appropriate conditions. Projection is used to select the customer names.


\begin{equation}
    \begin{split}
       & P_1 \leftarrow  borrower.loan\_number = loan.loan\_number \\
       & P_2 \leftarrow  loan.branch\_name = "Downtown" \\
       & \bigpi_{borrower.customer\_name}  (\bigsigma_{P_1 \wedge \ P_2}(borrower \times loan))
    \end{split}
\end{equation}
 

\end{document}
